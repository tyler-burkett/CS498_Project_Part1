\documentclass[a4paper,12pt]{article}
\usepackage{hyperref}
\usepackage{enumitem}

\begin{document}

\title{Project Requirements}
\author{ Tyler Burkett \\
	\and
	Kelsey Cole
	\and
	Cameron Lydon
	}
\date{\today}
\maketitle

\section*{Abbreviations and Related Terms}
\begin{itemize}
\item ABET - Accreditation Board for Engineering and Technology
\item DSU - Dean of Undergraduate Studies
\item SLO - Student Learning Outcomes
\item ADA - Americans with Disabilities Act
\item FERPA - Family Education Rights and Privacy Act 
\item PPRA - Protection of Pupil Rights Amendment
\end{itemize}

% \
\section*{Business Requirements}
\begin{enumerate}[label=BR\arabic*., ref=BR\arabic*]
% \item The client wants
\item The client wants to allow professors to be able to upload SLO evaluation information to a website
\item The client wants to increase the number of professors reporting SLO evaluations
\item The client wants to disseminate guides and informational documents about SLO evaluations to professors
\item The client wants to correct false assertions about SLO evaluations 
	\begin{enumerate}
	\item The client wants to emphasize that SLO evaluations are not grades
	\item The client wants to emphasize that SLO evaluations are not weighted (i.e. meeting---not exceeding---an SLO is expected)
	\item The client wants to emphasize that SLO evaluations are not evaluations of professors
	\end{enumerate}
\item The client wants to make maintaining and creating documentation for ABET accreditation more streamlined
\item The client wants to meet ADA compliance to ensure the system is accessible to all users
\item The client wants to meet FERPA compliance to ensure the system protects sensitive information
\item The client wants to meet PPRA compliance to ensure the system protects sensitive information
\item The client wants to have a class portfolio that contains a class score for the SLO evaluations (computed as the average of the student learning outcome scores)
\item The client wants to have a class portfolio file that contains all of the necessary student learning outcome information
	\begin{enumerate}
		\item Nameless student submissions with rubric and evaluation
		\item Number of students in each category (exceeds, meets, partially, not)
		\item An artifact score (computed as the \% of students who meet or exceeded the expectations for that artifact)
		\item A student learning outcome score (computed as the average of the three artifact scores)
	\end{enumerate}
\end{enumerate}

\section*{Functional Requirements}
\begin{enumerate}[label=FR\arabic*.]
% \item The system shall
\item The system shall create a form for professors to input information for SLO evaluations
\item The system shall output a zip file for each class portfolio after completion of SLO evaluations
\item The system shall allow an administrator user to assign SLOs to other users
\item The system shall mark course portfolios as read-only after a predefined period of time
\item The system shall display messages about SLO evaluation information as described in business requirements BR4
\item The system shall verify that the minimum number of artifacts has been uploaded for each SLO
\item The system shall display messages to confirm successful entry of SLO evaluations 
\item The system shall include a link to user documentation about the system at all times
\item The system shall include link for guides and informational documents about SLO evaluations
\item The system shall compute an artifact score for each artifact  based on the percentage of of students who meet or exceed the expectations of said artifact
\item The system shall compute a learning outcome score for each set of artifacts based on the average of the artifacts given for a SLO
\item The system shall compute a class score for a class portfolio based on the average SLO scores
\end{enumerate}

\section*{Non-Functional Requirements}
\begin{enumerate}[label=NFR\arabic*.]
% \item The system should 
\item The system should not ask for any personal identifying information during SLO evaluation entry
\item The system should not include any personal identifying information in the zip output
\item The system should contain at least three evaluations per SLO before output can be given
\item The system should randomly select a group of students for student evaluations with a maximum size of 20\% or 10 students for SLO evaluations
\item The system should handle login through linkblue
\item The system will use a MySQL database to store SLO evaluations
\item The system will use text that has a color contrast ratio of 4.5 or greater to the background
\item The system will specify alternate text for images
\item The system will allow clickable elements to be accessible by a keyboard
\item The system will allow users to upload documents to accompany artifacts in a class portfolio 
\end{enumerate}

\section*{User Assumptions}
\begin{itemize}
\item An artifact is a single evaluation of a single assignment that the students have been assigned and completed
\item Wait period for marking course portfolios as read-only is defined/adjusted by an administrator user of the system 
\item Professors do not have to do random selection of students for SLO evaluations
\item Professors do not have to manually enter SLO evaluation descriptions; an administrator user will input those for the professors to select from
\end{itemize}

\setlength{\tabcolsep}{10pt}
\renewcommand{\arraystretch}{1.5}
\begin{tabular}{ |c|c|c| }
 \hline
 \multicolumn{3}{|c|}{Requirements Traceability Matrix} \\
 \hline
 Business Requirements & Functional Requirements & Non-Functional Requirements\\
 \hline
 BR1 & FR1 & \\
 \hline
 BR2 & FR5 & \\
 \hline
 BR3 & FR5 & \\
     & FR8 & \\
     & FR9 & \\
\hline
 BR4 & FR5 & \\
 \hline
 BR5 & FR1 & NFR3\\
     & FR2 & NFR4\\
     & FR3 & \\
     & FR6 & \\
     & FR7 & \\
\hline
 BR6 &     & NFR7\\
     &     & NFR8\\
     &     & NFR9\\
\hline
 BR7 &     & NFR1\\
     &     & NFR2\\
\hline
 BR8 &     & NFR1\\
     &     & NFR2\\
\hline
 BR9 & FR2 & \\
      & FR10& \\
 \hline
 BR10 & FR2 & NFR10 \\
      & FR11& \\
      & FR12& \\
 \hline
\end{tabular}

\end{document}